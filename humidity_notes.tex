\chapter{Humidity}

\section{Humidity}

Humidity is a measure of the amount of water vapour contained in the air.

We can specify the humidity in a number of ways.
Humidity is linked to temperature, which makes understanding its measurement a bit harder.

\subsection{Relative humidity}

\href{https://en.wikipedia.org/wiki/Relative_humidity}{Relative humidity} quantifies the amount of water vapour in the air as a percentage \SIrange{0}{100}{\percent} of the maximum amount of water vapour the air can hold at a given temperature.

\subsection{Dew point}

The \href{https://en.wikipedia.org/wiki/Dew_point}{dew point} is the temperature at which water vapour liquifies.

When air is at \SI{100}{\percent} relative humidity, the dew point is equal to its temperature.
It is said to be saturated. 

\subsection{Calculation}

Magnus formula: 
\begin{align}
  y (T, RH) & = \ln \left ( \frac{RH}{100} \right ) + \frac{bT}{c+T} \\
  T_{\mbox{dp}} & = \frac{c \cdot y (T, RH)}{b - y(T, RH)}
\end{align}
where $b = 18.678$ and $c=257.14$ C. 

Simplified approximation:
\begin{align}
  T_{\mbox{dp}} & \approx T - \frac{100-\mbox{RH}}{5} \\
  \Rightarrow \mbox{RH} & = 100 - 5 ( T - T_{\mbox{dp}} ) 
\end{align}


\section{Humidity envelope}

\subsection{Environmental classes}

The ASHRAE environmental classes require humidity to be controlled.

\autoimage{ashrae_environmental_classes}{ASHRAE environmental classes}{ashrae-environmental-classes}


\subsection{Humidity control}

Precision air conditioning systems manage humidity along with temperature.

For the purposes of humidity control we will assume a DX CRAC and Chilled-Water CRAH to be identical.

\section{Dehumidification}
\label{sec:dehumidification}

\subsection{Dehumidification during cooling}
\label{sec:dehumidification-during-cooling}

Dehumidification will occur naturally as a byproduct of cooling.
When air is cooled below its dew point as it passes through the evaporator, some of the moisture will condense.
The condensate drains or is pumped away to waste.

The degree of dehumidification can be controlled somewhat by varying the evaporator fan speed and the temperature of the evaporator coil:
\begin{itemize}
\item
  The longer the air spends in contact with the coil, the more dehumidification will occur.
  Dehumidification performance should be better at lower fan speeds (with all other variables being equal.)
\item
  The colder the evaporator coil, the more dehumificiation will occur.
  Some systems with an electronically controlled expansion valve can vary this to modulate the dehumidification performance.
\end{itemize}

\subsection{Dehumidification with reheat}
\label{sec:dehumidification-with-reheat}

Sometimes we will have a demand for dehumidification when cooling is not also required.
We can often tolerate a small degree of overcooling.
Other times we will need to re-heat the air:
\begin{itemize}
\item Electric re-heat where electric heaters warm the air after it passes over the evaporator.
  These are sometimes modulated using a device called a Silicon Controlled Rectifier to give precision control of the air temperature.
\item Hot water or steam re-heat where an additional reheat coil is situated downstream of the evaporator.
  Controlled by a motorised valve that can modulate the hot water/steam flow on/off to maintain the air temperature.
  Requires source of hot water such as a central boiler, often shared with other services like space or water heating. 
\item Hot gas reheat where the refrigeration system itself releases heat back into the airstream downstream of the evaporator.
  Requires less energy than the electric or hot water re-heat options.
\end{itemize}

\subsection{Independent dehumidifier}
\label{sec:independent-dehumidifier}

Independent dehumifiers are like a CRAC except that both the evaporator and condenser coil are in the indoor airstream.
They are used in situations where dehumidification is required without cooling, and are not normally seen in data centre environments. 

The evaporator cools air below its dew point and the condenser coil re-heats it. 

\section{Humidification}
\label{sec:humidification}

Humidification involves adding moisture to the air, increasing its relative humidity.
Humidifiers can be separate independent devices but are commonly incorporated with a CRAC.

\subsection{Infrared humidifier}
\label{sec:infrared-humidifier}

The infrared humidifier uses quartz lamps to boil water in a pan, \autoref{fig:crac-infrared-humidifier-photo}.

\autoimage{crac_infrared_humidifier_photo}{Photo of CRAC with infrared humidifier}{crac-infrared-humidifier-photo}

Unwanted heat will be introduced to the controlled space by an infrared humidifier.
This is limited if only a small amount of humidification is required (to offset dehumidification caused by cooling).

\subsection{Steam canister humidifier}
\label{sec:steam-canister-humidifier}

Steam canister humifiers use electrodes to pass a current through water, heating it in the process, \autoref{fig:steam-canister-humidifier-schematic}.
The steam boiled off is released into the air.

\autoimage{steam_canister_humidifier_schematic}{Schematic diagram of steam canister humidifier}{steam-canister-humidifier-schematic}

Similar to the infrared humidifier, the steam humidifier will introduce unwanted heat into the controlled space.
This can be kept relatively low if the humidifier's duty cycle is low. 

\subsection{Ultrasonic humidifier}
\label{sec:ultrasonic-humidifier}

Ultrasonic humidifiers atomise water into tiny droplets, releasing it into the air.

Ultrasonic humidifiers are much cheaper to run than infrared and steam humidifiers and do not introduce unwanted heat into the controlled space.
They are available as retrofit kits for many CRACs, with a similar form factor to infrared units, \autoref{fig:ultrasonic-humidifier-photo}.

\autoimage{ultrasonic_humidifier_photo}{Photo of an ultrasonic humidifier designed for retrofit applications}{ultrasonic-humidifier-photo}

Ultrasonic humidifiers need a very clean water supply which adds to their overall cost, but is economic in multiple unit installations. 





